% --- -----------------------------------------------------------------
% --- Elementos usados na Capa e na Folha de Rosto.
% --- EXPRESS�ES ENTRE <> DEVER�O SER COMPLETADAS COM A INFORMA��O ESPEC�FICA DO TRABALHO
% --- E OS S�MBOLOS <> DEVEM SER RETIRADOS 
% --- -----------------------------------------------------------------
\autor{Luan Teylo Gouveia Lima} % deve ser escrito em maiusculo

\titulo{Escalonamento de Tarefas e Alocação de Arquivos de Dados de \textit{Workflows} Científicos em Nuvens Computacionais}

\instituicao{UNIVERSIDADE FEDERAL FLUMINENSE}

\orientador{Lúcia Maria de Assumpção Drummond}

\coorientador{Yuri Abitbol de Menezes Frota} % se nao existir co-orientador apague essa linha

\local{NITER\'{O}I}

\data{2017} % ano da defesa

\comentario{Dissertação de mestrado apresentada ao Programa de Pós-Graduação em Computação da Universidade Federal Fluminense como requisito parcial para a obtenção do Grau de Mestre em Computação. Área de concentração: Sistemas de Computação} %preencha com a sua area de concentracao


% --- -----------------------------------------------------------------
% --- Capa. (Capa externa, aquela com as letrinhas douradas)(Obrigatorio)
% --- ----------------------------------------------------------------
\capa

% --- -----------------------------------------------------------------
% --- Folha de rosto. (Obrigatorio)
% --- ----------------------------------------------------------------
\folhaderosto


\pagestyle{ruledheader}
\setcounter{page}{1}
\pagenumbering{roman}

% --- -----------------------------------------------------------------
% --- Termo de aprovacao. (Obrigatorio)
% --- ----------------------------------------------------------------
\cleardoublepage
\thispagestyle{empty}

\vspace{-60mm}

\begin{center}
   {\large Luan Teylo Gouveia Lima}\\
   \vspace{7mm}
   
   Escalonamento de Tarefas e Alocação de Arquivos de Dados de \textit{Workflows} Científicos em Nuvens Computacionais\\
  \vspace{10mm}
\end{center}

\noindent
\begin{flushright}
\begin{minipage}[t]{8cm}

Dissertação de Mestrado apresentada ao Programa de Pós-Graduação em Computação da Universidade Federal Fluminense como requisito parcial para a obtenção do Grau de Mestre em Computação. Área de concentração: Sistemas de Computação %preencha com a sua area de concentracao

\end{minipage}
\end{flushright}
\vspace{1.0 cm}
\noindent
Aprovada em 17 de março 2017. \\
\begin{flushright}
  \parbox{11cm}
  {
  \begin{center}
  BANCA EXAMINADORA \\
  \vspace{6mm}
  \rule{11cm}{.1mm} \\
    Profa. Lúcia M. A. Drummond - Orientador, IC-UFF \\
    \vspace{6mm}
  \rule{11cm}{.1mm} \\
    Prof. Yuri Abitbol de Menezes Frota - Co-orientador, IC-UFF \\
    \vspace{6mm}
  \rule{11cm}{.1mm} \\
    Prof. Daniel Cardoso Moraes de Oliveira, IC-UFF\\
    \vspace{6mm}
  \rule{11cm}{.1mm} \\
    Profa. Cristiana Barbosa Bentes, UERJ\\
  \vspace{6mm}
  \end{center}
  }
\end{flushright}
\begin{center}
  \vspace{4mm}
  Niter\'{o}i \\
  %\vspace{6mm}
  2017

\end{center}

% --- -----------------------------------------------------------------
% --- Dedicatoria.(Opcional)
% --- -----------------------------------------------------------------
\cleardoublepage
\thispagestyle{empty}
\vspace*{200mm}

\begin{flushright}
{\em 
À minha família.
}
\end{flushright}
\newpage


% --- -----------------------------------------------------------------
% --- Agradecimentos.(Opcional)
% --- -----------------------------------------------------------------
\pretextualchapter{Agradecimentos}
\hspace{5mm}
% Elemento opcional, colocado ap�s a dedicat�ria (ABNT, 2005). 

À professora Lúcia pela orientação, e por toda a confiança depositada na elaboração deste trabalho. Sou grato pela oportunidade de trabalhar com uma pessoa tão comprometida com a orientação de seus alunos e com os resultados de seu trabalho.

Ao professor Yuri pela co-orientação, ajuda na elaboração e explicação de várias ideias que fizeram deste um trabalho no qual me orgulho muito. Sem a sua ajuda e o seu bom humor, o meu desempenho com certeza não seria o mesmo.

Ao professor Daniel por toda ajuda e pela paciência em suas explicações extremamente claras. Sou grato pela dedicação que você tem com os seus alunos, e por sempre ter arrumado tempo para esclarecer as minhas dúvidas.

Ao Ubiratam, sem o qual uma parte importantíssima deste trabalho não teria a mesma qualidade. Sou muito grato por poder contar com a sua experiência e dedicação.

À Pamela, com quem compartilho as minhas conquistas. Obrigado pelo relacionamento duradouro e construtivo.

Aos meus companheiros de laboratório Leonardo, Maicon e Rodrigo, que entre cafés e conversas, me proporcionaram um ambiente de trabalho rico e prazeroso. 

A todos os professores e colegas que de alguma forma colaboraram para a minha formação.

Ao CNPq, pela bolsa de mestrado concedida.
% --- -----------------------------------------------------------------
% --- Resumo em portugues.(Obrigatorio)
% --- -----------------------------------------------------------------
\begin{resumo}


Na última década, um número crescente de experimentos computacionalmente intensivos envolvendo grandes volumes de dados têm sido modelados na forma de \textit{workflows} científicos. Ao mesmo tempo, as nuvens computacionais surgem como um ambiente promissor para executar esse tipo de aplicação. Neste cenário, a investigação de estratégias de escalonamentos se tornaram essenciais, sendo este um campo de pesquisa extremamente popular. No entanto, poucos trabalhos consideram o problema da alocação de dados durante a resolução do problema de escalonamento de tarefas. 

Um \textit{workflow} é geralmente representado como um grafo, no qual os nós equivale as tarefas e, nestes casos, o problema de escalonamento consiste em alocar essas tarefas a máquinas que as executarão em um tempo pré definido.  O objetivo é reduzir o tempo total de execução de todo o \textit{workflow}.

Neste trabalho é mostrado que o escalonamento de \textit{workflows} científicos pode ser melhorado quando o problema de escalonamento de tarefa e alocação de dados são tratados de forma conjunta. Para isso, uma nova representação, na qual os nós do grafo representam tanto tarefas como dados, é proposta. Além disso, o problema de Escalonamento de Tarefas e Alocação de Dados é definido, considerando esse novo modelo. Esse problema foi formulado como um problema de programação inteira. Por fim, um algoritmo evolucionário híbrido capaz de escalonar tarefas e alocar os dados em ambientes de nuvens computacionais também é apresentado.

% Elemento obrigat�rio, constitu�do de uma sequ�ncia de frases concisas e objetivas e n�o de uma simples enumera��o de t�picos, n�o ultrapassando 500 palavras (ABNT, 2005).

{\hspace{-8mm} \bf{Palavras-chave}}: Problema de escalonamento, Alocação de Dados, \textit{Workflow} Científico, Metaheurística.

\end{resumo}

% --- -----------------------------------------------------------------
% --- Resumo em lingua estrangeira.(Obrigatorio)
% --- -----------------------------------------------------------------
\begin{abstract}

% Elemento obrigatório, em língua estrangeira, com as mesmas caracter�sticas do resumo em l�ngua vern�cula (ABNT, 2005).

A growing number of data- and compute-intensive experiments have been modeled as scientific workflows in the last decade. Meanwhile, clouds have emerged as a prominent environment to execute this type of applications.  In this scenario, the investigation of workflow scheduling strategies, aiming at reducing its execution times, became a top priority and a very popular research field.  
However, few works consider the problem of data file assignment when solving the task scheduling problem.  Usually, a workflow is represented by a graph where nodes represent tasks and the scheduling problem consists in allocating tasks to machines to be executed at a predefined time aiming at reducing the makespan of the whole workflow.  

In this work, we show that the scheduling  of scientific workflows can be improved when both task scheduling and the data file assignment problems are treated together. Thus, we propose a new workflow representation, where nodes of the workflow graph represent either tasks or data files, and define the Task Scheduling and Data Allocation Problem, considering this new model. We formulated this problem as an integer programming problem. Moreover, a hybrid evolutionary algorithm for solving it is also introduced.


{\hspace{-8mm} \bf{Keywords}}: Scheduling Problem, Data Allocation, Scientific Workflow, Metaheuristic.

\end{abstract}

% --- -----------------------------------------------------------------
% --- Lista de figuras.(Opcional)
% --- -----------------------------------------------------------------
%\cleardoublepage
\listoffigures


% --- -----------------------------------------------------------------
% --- Lista de tabelas.(Opcional)
% --- -----------------------------------------------------------------
\cleardoublepage
%\label{pag:last_page_introduction}
\listoftables
\cleardoublepage

% --- -----------------------------------------------------------------
% --- Lista de abreviatura.(Opcional)
%Elemento opcional, que consiste na rela��o alfab�tica das abreviaturas e siglas utilizadas no texto, seguidas das %palavras ou express�es correspondentes grafadas por extenso. Recomenda-se a elabora��o de lista pr�pria para cada %tipo (ABNT, 2005).
% --- ----------------------------------------------------------------
\cleardoublepage
\pretextualchapter{Lista de Abreviaturas e Siglas}
\begin{tabular}{lcl}
WfC & : & \textit{Workflow} Científico;\\
SGWfC & : & Sistema de Gerenciamento de \textit{Workflow} Científico;\\
HPC & : & \textit{High Performance Computing};\\
MV & : & Máquina Virtual;\\
DAG& : & \textit{Directed Acyclic Graph};\\
PaaS & : &\textit{Platform as a Service};\\
SaaS & : & \textit{Software as a Service};\\
IaaS & : & \textit{Infrastructure as a Service};\\
ETAA & : & Escalonamento de Tarefas e Alocação de Arquivos;\\
AEH-ETAA & : & Algoritmo Evolutivo Híbrido para Escalonamento de\\ 
            && Tarefas e Alocação de Arquivos;
        
\end{tabular}
% --- -----------------------------------------------------------------
% --- Sumario.(Obrigatorio)
% --- -----------------------------------------------------------------
\pagestyle{ruledheader}
\tableofcontents


